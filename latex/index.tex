\subsection*{Introduction}

このプログラムは遺伝的アルゴリズム(\+Genetic Algorithm)の\+C++による実装である。\+C++17の使用を前提としている。

\subsection*{Previous Studies}

\subsubsection*{遺伝的アルゴリズムについて}

Kazuhide Okamura, 遺伝的アルゴリズム(\+Genetic Algorithm)を始めよう!, \href{https://www.slideshare.net/kzokm/genetic-algorithm-41617242}{\texttt{ https\+://www.\+slideshare.\+net/kzokm/genetic-\/algorithm-\/41617242}}

\subsubsection*{設計について}

@nanasess, ログ設計指針, \href{https://qiita.com/nanasess/items/350e59b29cceb2f122b3}{\texttt{ https\+://qiita.\+com/nanasess/items/350e59b29cceb2f122b3}}

\subsection*{Release Note}

\tabulinesep=1mm
\begin{longtabu}spread 0pt [c]{*{3}{|X[-1]}|}
\hline
\PBS\raggedleft \cellcolor{\tableheadbgcolor}\textbf{ 日付  }&\PBS\raggedleft \cellcolor{\tableheadbgcolor}\textbf{ 版数  }&\PBS\centering \cellcolor{\tableheadbgcolor}\textbf{ 備考   }\\\cline{1-3}
\endfirsthead
\hline
\endfoot
\hline
\PBS\raggedleft \cellcolor{\tableheadbgcolor}\textbf{ 日付  }&\PBS\raggedleft \cellcolor{\tableheadbgcolor}\textbf{ 版数  }&\PBS\centering \cellcolor{\tableheadbgcolor}\textbf{ 備考   }\\\cline{1-3}
\endhead
\PBS\raggedleft March 18th 2019  &\PBS\raggedleft 0.\+0  &アップロード   \\\cline{1-3}
\end{longtabu}
