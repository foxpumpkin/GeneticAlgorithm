下記の5つに分類できる。

\subsubsection*{main()}

Genetic\+Algorithmライブラリの動作確認用テストプログラムである。

\subsubsection*{Genetic\+Algorithm}

本リポジトリのメインとなる遺伝的アルゴリズムを実装する。

\subsubsection*{Log}

ログ出力を管理する。ベースは\+Log\+System classであるが、 Log classでラップするように運用する。ただし、\+Log\+System classは Singletonで設計/実装されているため、必ずしも\+Log classを使用する必要はない。 非推奨であるが、直接\+Log\+System classを呼んでログ出力(統計)も可能である。

\subsubsection*{Option}

コマンドライン引数を解釈する。 boost\+::program\+\_\+optionsを使用し、コマンドライン引数を解釈する。 現時点(\+March 2019)では一般化されていないため、他のプログラムに 流用するのは難しいと思われる。

\subsubsection*{Define}

全体で使用する変数やクラスを記述する。

\paragraph*{名前空間 global}

プログラム全体で使用するフラグ類である。

\paragraph*{名前空間 loglevel}

ログレベルを表すuint64\+\_\+t型の変数を宣言/定義し、 また、loglevel\+::to\+\_\+string()を使用し、uint64\+\_\+t型のログレベルを引数としてstring型へ 変換できる。

\paragraph*{名前空間 color}

標準出力/標準エラー出力時の色を指定することを目的とした変数およびクラスを宣言/定義する。 色の指定はuint64\+\_\+t型で宣言/定義される変数を使用する。

color\+::to\+\_\+string$<$color\+::\+Color$>$()で引数として与えるstringに対してglobal\+::color\+\_\+logが trueの場合はカラータグを付与したstringを返す。 global\+::color\+\_\+logがfalseの場合はタグを付与せずそのまま返す。 